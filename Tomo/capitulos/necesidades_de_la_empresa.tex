\chapter{Necesidades de la Empresa}
En addsolutions la innovación es la principal clave para hacer más eficiente los procesos de recursos humanos y para ello se han desarrollado soluciones tecnológicas de avanzada que simplifican la administración del talento humano. Hoy en día hay una limitación en software interno con respecto a instrumentos/estudios, actualmente se cuenta con sistemas que permiten construir instrumentos enmarcados en un único proceso o área y las necesidades de nuestros clientes han crecido, han surgido nuevos requerimientos que demandan la realización de estudios más complejos que incluyen la realización de encuestas, pruebas, cuestionarios, test y/o instrumentos de medición, percepción y opiniones sobre aspectos relevantes a su organización.

El desarrollo e implementación de estos productos han permitido a addsolutions constatar que la innovación, profundidad de análisis y la calidad de atención son los principales elementos de su sistema de apoyo a los clientes. Esta fórmula les ha permitido en corto plazo servir a clientes muy importantes tanto nacionales como internacionales. Sin embargo estos sistemas han estado enmarcados en áreas y procesos muy específicos, trayendo como consecuencia limitaciones al momento de cubrir proyectos de otras áreas del mundo empresarial.

Para cubrir esta inflexibilidad, en un proyecto reciente titulado “Estudio de Beneficios de la Industria Farmacéutica” que está dirigido a empresas en México y contempla las etapas de investigación, construcción de instrumentos, recolección de datos, procesamiento y presentación de resultados se utilizó un servicio de encuestas online llamado “Free Online Surveys”. Con este servicio se definieron 19 instrumentos en donde participaron 33 empresas por cada uno, definiendo tipos de preguntas complejas y no contempladas en ninguno de los sistemas desarrollados internamente. El uso de esta herramienta implica que las etapas de procesamiento, extracción y presentación de resultados sean completamente manuales, en donde se usan programas como Excel ya que no se cuenta con acceso a la base de datos del servicio online. Así como el proyecto antes descrito existen muchos más en donde se presentan limitaciones con respecto a la construcción de instrumentos, accesibilidad de datos y niveles de complejidad en los estudios; ante esta necesidad  se plantea el desarrollo de una aplicación colaborativa e interactiva en ambiente Web, que permita a Empresas la aplicación de encuestas, pruebas, cuestionarios, test y/o instrumentos de medición de percepción u opiniones sobre aspectos relevantes para la organización. La aplicación web a desarrollar no estará diseñada para ser usada solo por recursos humanos sino por cualquier área.
